\documentclass[11pt]{article}
\usepackage[margin=1in]{geometry}
\usepackage{hyperref, listings}
\usepackage{tipa}
\usepackage[shortlabels]{enumitem}
\setlist{  
  listparindent=\parindent,
  parsep=0pt,
}

\lstset{
    basicstyle=\ttfamily
}

\begin{document}
 
\title{\textbf{NPRG005 Project Proposal: Alonzo}}
\author{Marcel Goh}
\maketitle

Alonzo is an interactive interpreter that performs normal-order $\beta$-reduction on expressions in the $\lambda$-calculus. Users will be able to interact with Alonzo via a read-eval-print loop where names can be assigned to $\lambda$-terms. The end result will look something like this:

\begin{lstlisting}
    ]=> TRUE = \p.\q.p
    TRUE = \p.\q.p
    ]=> FALSE := \p.\q.q
    FALSE = \p.\q.q
    ]=> AND = \p.\q.p q p
    AND = \p.\q.p q p
    ]=> AND TRUE FALSE
    \p.\q.q (FALSE)
\end{lstlisting}

Alonzo stores the de Bruijn indices of the terms it has seen, along with the names they are bound to. After evaluating a term, it checks to see if it has seen the term before. This is how it knows in the above example that \texttt{AND TRUE FALSE} evaluates to \texttt{FALSE}.

\end{document}
